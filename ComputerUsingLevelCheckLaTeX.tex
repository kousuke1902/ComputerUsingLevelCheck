\documentclass[lualatex]{jlreq}
\usepackage{graphicx}

\usepackage{fancyhdr} 

\title{電子計算機習得技能確認表}
\author{オカン教育課 コウスケ}
\date{2021年4月21日}

\begin{document}
\maketitle

\thispagestyle{fancy} 
\lhead{企業秘密っぽい資料} %左側ヘッダの定義 
\cfoot{} %中央フッターの定義 

\begin{center}
記載されている技能に対して自身が該当する項目欄にチェックをしていくこと.
\end{center}

\begin{table}[h]
\begin{tabular}{|c|c|c|c|}
\hline
 & 知っとる & 知らん & 覚えたった \\ \hline
クリック, ダブルクリックができる &  &  &  \\ \hline
ドラッグ, ドロップができる &  &  &  \\ \hline
右クリックでどんなことができるか理解している &  &  &  \\ \hline
エクスプローラーを開ける &  &  &  \\ \hline
ファイルを開ける &  &  &  \\ \hline
フォルダを開ける &  &  &  \\ \hline
ローマ字を理解している &  &  &  \\ \hline
英字を入力できる &  &  &  \\ \hline
平仮名, 片仮名を入力できる &  &  &  \\ \hline
数字を入力できる &  &  &  \\ \hline
半角片仮名を入力できる &  &  &  \\ \hline
コピー, 切り取り, ペーストができる &  &  &  \\ \hline
ブラウザを開ける &  &  &  \\ \hline
検索エンジンを用いて検索できる &  &  &  \\ \hline
シャットダウン, ログオフ (ログアウト), 再起動を使い分けられる &  &  &  \\ \hline
\end{tabular}
\end{table}
\end{document}
